\documentclass[
  % all of the below options are optional and can be left out
  % course name (default: 2IL50 Data Structures)
  course = {{16-720B Computer Vision}},
  % quartile (default: 3)
  quartile = {{1}},
  % assignment number/name (default: 1)
  assignment = 6\ -\ Object\ Tracking\ in\ Videos,
  % student name (default: Some One)
  name = {{Kangle Deng}},
  % student number, NOT S-number (default: 0123456)
  % studentnumber = {{0123456 ; 0314159}},
  % student email (default: s.one@student.tue.nl)
  email = {{kangled@andrew.cmu.edu}},
  % first exercise number (default: 1)
  firstexercise = 1
]{aga-homework}

\usepackage{url}
\usepackage{subfigure}
\usepackage{xcolor}  	%高亮使用的颜色
\definecolor{commentcolor}{RGB}{85,139,78}
\definecolor{stringcolor}{RGB}{206,145,108}
\definecolor{keywordcolor}{RGB}{34,34,250}
\definecolor{backcolor}{RGB}{220,220,220}
\usepackage{accsupp}	
\newcommand{\emptyaccsupp}[1]{\BeginAccSupp{ActualText={}}#1\EndAccSupp{}}

\usepackage{listings}
\lstset{						%高亮代码设置
	language=python, 					%Python语法高亮
	linewidth=0.9\linewidth,      		%列表list宽度
	%basicstyle=\ttfamily,				%tt无法显示空格
	commentstyle=\color{commentcolor},	%注释颜色
	keywordstyle=\color{keywordcolor},	%关键词颜色
	stringstyle=\color{stringcolor},	%字符串颜色
	%showspaces=true,					%显示空格
	numbers=left,						%行数显示在左侧
	numberstyle=\tiny\emptyaccsupp,		%行数数字格式
	numbersep=5pt,						%数字间隔
	frame=single,						%加框
	framerule=0pt,						%不划线
	escapeinside=@@,					%逃逸标志
	emptylines=1,						%
	xleftmargin=3em,					%list左边距
	backgroundcolor=\color{backcolor},	%列表背景色
	tabsize=4,							%制表符长度为4个字符
	% gobble=4,							%忽略每行代码前4个字符
	basicstyle          =   \ttfamily,          % 基本代码风格
    keywordstyle        =   \color{blue},         % 关键字风格
    commentstyle        =   \rmfamily\itshape,  % 注释的风格,斜体
    stringstyle         =   \ttfamily,  % 字符串风格
    breaklines      =   true,   % 自动换行,建议不要写太长的行
	}

\begin{document}
\section{Theory Questions}
\noindent\textbf{Q1.1: Calculating the Jacobian}

We have
\begin{equation*}
    \begin{aligned}
    u' = (1+p_1)u + p_3v + p_5, \\
    v' = p_2u + (1+p_4)v + p_6.
    \end{aligned}
\end{equation*}

So
\begin{equation*}
    \frac{\partial \mathbf{W}}{\partial \mathbf{p}} = \left(\begin{array}{cccccc}
       u  & 0 & v & 0 & 1 & 0 \\
       0  & u & 0 & v & 0 & 1
    \end{array}\right).
\end{equation*}

\noindent\textbf{Q1.2: Computational complexity Lucas Kanade}

We analyze the complexity of Algorithm 1 step by step:
\begin{itemize}
    \item (2) Iterate $\mathbf{x}$ through the range of Template coordinates. Each warp requires $O(p)$ computation. So the complexity should be $O(np)$.
    \item (3) Iterate $\mathbf{x}$ through the range of Template coordinates. So the complexity should be $O(n)$.
    \item (4) The gradient $\nabla \mathbf{I}$ can be pre-calculated out of the loop with complexity $O(m)$. Iterate $\mathbf{x}$ through the range of Template coordinates. So the complexity should be $O(n)$.
    \item (5) Iterate $\mathbf{x}$ through the range of Template coordinates. For each $\mathbf{x}$, evaluating the Jacobian requires $O(p)$. So the complexity should be $O(np)$.
    \item (6) Iterate $\mathbf{x}$ through the range of Template coordinates. For each $\mathbf{x}$, computing the steepest descent requires $O(p)$. So the complexity should be $O(np)$.
    \item (7) Iterate $\mathbf{x}$ through the range of Template coordinates. For each $\mathbf{x}$, calculation requires $O(p^2)$. So the complexity should be $O(np^2)$.
    \item (8) The summation requires $O(np)$. And the matrix multiplication requires $O(p^3)$. Assuming $p^3 << n$, the complexity should be $O(np)$.
    \item (9) Updating $p$ requires $O(p)$.
\end{itemize}

In conclusion, the total complexity of one runtime iteration is $O(np^2)$.

\noindent\textbf{Q1.3: Computational complexity Matthews-Baker}

We analyze the complexity of pre-computation in Algorithm 2 step by step:
\begin{itemize}
    \item (1) Evaluating $\nabla \mathbf{T}$ requires $O(n)$.
    \item (2) Iterate $\mathbf{x}$ through the range of Template coordinates. For each $\mathbf{x}$, evaluating the Jacobian requires $O(p)$. So the complexity should be $O(np)$.
    \item (3) Iterate $\mathbf{x}$ through the range of Template coordinates. For each $\mathbf{x}$, computing the steepest descent requires $O(p)$. So the complexity should be $O(np)$.
    \item (4) The matrix multiplication requires $O(np^2)$.
\end{itemize}

So the total complexity of the pre-computation is $O(np^2)$.

We analyze the complexity of each runtime iteration step by step:
\begin{itemize}
    \item (7) Iterate $\mathbf{x}$ through the range of Template coordinates. Each warp requires $O(p)$ computation. So the complexity should be $O(np)$.
    \item (8) Iterate $\mathbf{x}$ through the range of Template coordinates. So the complexity should be $O(n)$.
    \item (9) The summation requires $O(np)$. And the matrix multiplication requires $O(p^3)$. Assuming $p^3 << n$, the complexity should be $O(np)$.
    \item (10) Updating $p$ requires $O(p)$.
\end{itemize}

So the complexity of each runtime iteration step is $O(np)$. This is less than the run time of the regular Lucas-Kanade method.

\section{Implementing the Trackers}
\noindent\textbf{Q2.1:} 
The code is as below:
\begin{lstlisting}
def LucasKanade(It, It1, rect):
    # Input: 
    #   It: template image
    #   It1: Current image
    #   rect: Current position of the object
    #   (top left, bot right coordinates: x1, y1, x2, y2)
    # Output:
    #   p: movement vector dx, dy
    
    # set up the threshold
    threshold = 0.01875
    maxIters = 100
    p = np.zeros(2)          
    x1,y1,x2,y2 = rect

    # put your implementation here
    H, W = It.shape
    tH, tW = int(y2 - y1), int(x2 - x1)
    iter_cnt = 0

    It1Spline = RectBivariateSpline(np.arange(H), np.arange(W), It1)
    ItSpline = RectBivariateSpline(np.arange(H), np.arange(W), It)
    T = ItSpline(np.arange(tH)+y1, np.arange(tW)+x1)

    Iy, Ix = np.gradient(It1)
    IySpline = RectBivariateSpline(np.arange(H), np.arange(W), Iy)
    IxSpline = RectBivariateSpline(np.arange(H), np.arange(W), Ix)

    while True:
        dx, dy = p
        y, x = np.arange(tH), np.arange(tW)
        y = y + y1 + dy
        x = x + x1 + dx
        imWarp = It1Spline(y, x)
        gradWarp = np.stack([IySpline(y,x), IxSpline(y,x)], axis=-1)
        imerr = T - imWarp
        A = gradWarp.reshape(-1, 2)
        b = imerr.reshape(-1)
        dp = np.linalg.lstsq(A, b, rcond=None)[0]
        p = p + dp[::-1]
        iter_cnt += 1
        if np.linalg.norm(dp) <= threshold or iter_cnt > maxIters:
            break
    return p
\end{lstlisting}

\noindent\textbf{Q2.2:}
The code is as below:
\begin{lstlisting}
def LucasKanadeAffine(It, It1, rect):
    # Input: 
    #   It: template image
    #   It1: Current image
    #   rect: Current position of the object
    #   (top left, bot right coordinates: x1, y1, x2, y2)
    # Output:
    #   M: the Affine warp matrix [2x3 numpy array]

    # set up the threshold
    threshold = 0.01875
    maxIters = 100
    p = np.zeros((6,))
    x1,y1,x2,y2 = rect

    # put your implementation here
    H, W = It.shape
    tH, tW = int(y2 - y1), int(x2 - x1)
    iter_cnt = 0

    It1Spline = RectBivariateSpline(np.arange(H), np.arange(W), It1)
    ItSpline = RectBivariateSpline(np.arange(H), np.arange(W), It)
    T = ItSpline(np.arange(tH)+y1, np.arange(tW)+x1)

    Iy, Ix = np.gradient(It1)
    IySpline = RectBivariateSpline(np.arange(H), np.arange(W), Iy)
    IxSpline = RectBivariateSpline(np.arange(H), np.arange(W), Ix)

    y, x = np.arange(tH)+y1, np.arange(tW)+x1
    jacob = np.zeros((tH,tW,2,6), dtype=np.float32)
    jacob[:,:,0,4] = jacob[:,:,1,5] = 1
    jacob[:,:,0,0] = jacob[:,:,1,1] = y[:,None]
    jacob[:,:,0,2] = jacob[:,:,1,3] = x[None,:]

    y_grid, x_grid = np.meshgrid(y, x, indexing='ij')

    while True:
        y_w = (1+p[0]) * y_grid.flatten() + p[2] * x_grid.flatten() + p[4]
        x_w = p[1] * y_grid.flatten() + (1 + p[3]) * x_grid.flatten() + p[5]
        imWarp = It1Spline(y_w, x_w, grid=False).reshape(tH,tW)
        gradWarp = np.stack([IySpline(y_w,x_w, grid=False), IxSpline(y_w,x_w, grid=False)], axis=-1).reshape(tH,tW,2)
        imerr = T - imWarp
        
        A = gradWarp[:,:,0,None] * jacob[:,:,0,:] + gradWarp[:,:,1,None] * jacob[:,:,1,:]
        A = A.reshape(-1, 6)
        b = imerr.reshape(-1)
        dp = np.linalg.lstsq(A, b, rcond=None)[0]
        p = p + dp
        iter_cnt += 1
        if np.linalg.norm(dp) <= threshold or iter_cnt > maxIters:
            break

    # reshape the output affine matrix
    M = np.array([[1.0+p[3], p[1], p[5]],
                [p[2], 1.0+p[0], p[4]]]).reshape(2, 3)

    return M
\end{lstlisting}

\noindent\textbf{Q2.3:}
The code is as below:
\begin{lstlisting}
def InverseCompositionAffine(It, It1, rect):
    # Input: 
    #   It: template image
    #   It1: Current image
    #   rect: Current position of the object
    #   (top left, bot right coordinates: x1, y1, x2, y2)
    # Output:
    #   M: the Affine warp matrix [2x3 numpy array]

    # set up the threshold
    threshold = 0.01875
    maxIters = 100
    p = np.zeros((6,1))
    x1,y1,x2,y2 = rect

    # put your implementation here
    H, W = It.shape
    tH, tW = int(y2 - y1), int(x2 - x1)
    iter_cnt = 0
    It1Spline = RectBivariateSpline(np.arange(H), np.arange(W), It1)
    ItSpline = RectBivariateSpline(np.arange(H), np.arange(W), It)
    T = ItSpline(np.arange(tH)+y1, np.arange(tW)+x1)

    Iy, Ix = np.gradient(T)
    y, x = np.arange(tH)+y1, np.arange(tW)+x1
    jacob = np.zeros((tH,tW,2,6), dtype=np.float32)
    jacob[:,:,0,4] = jacob[:,:,1,5] = 1
    jacob[:,:,0,0] = jacob[:,:,1,1] = y[:,None]
    jacob[:,:,0,2] = jacob[:,:,1,3] = x[None,:]
    A = Iy[:,:,None] * jacob[:,:,0,:] + Ix[:,:,None] * jacob[:,:,1,:]
    A = A.reshape(-1, 6)

    y_grid, x_grid = np.meshgrid(y, x, indexing='ij')

    while True:
        y_w = (1+p[0]) * y_grid.flatten() + p[2] * x_grid.flatten() + p[4]
        x_w = p[1] * y_grid.flatten() + (1 + p[3]) * x_grid.flatten() + p[5]
        imWarp = It1Spline(y_w, x_w, grid=False).reshape(tH,tW)
        imerr = imWarp - T

        b = imerr.reshape(-1)
        dp = np.linalg.lstsq(A, b, rcond=None)[0]
        dP = np.eye(3)
        dP[:2, :] += dp.reshape(3,2).T
        dP_inv = np.linalg.inv(dP)
        P_new = np.eye(3)
        P_new[:2, :] += p.reshape(3,2).T
        P_new = P_new.dot(dP_inv)
        p = (P_new - np.eye(3))[:2,:].T.flatten()
        iter_cnt += 1
        if np.linalg.norm(dp) <= threshold or iter_cnt > maxIters:
            break

    # reshape the output affine matrix
    M = np.array([[1.0+p[3], p[1], p[5]],
                [p[2], 1.0+p[0], p[4]]]).reshape(2, 3)

    return M
\end{lstlisting}

\noindent\textbf{Q2.4:} 


\end{document}
